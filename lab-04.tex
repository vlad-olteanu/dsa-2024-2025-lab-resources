\documentclass{exam}
\usepackage{amssymb}
\usepackage{lipsum} 
\usepackage{graphicx}
\usepackage[
backend=biber
]{biblatex}
\addbibresource{refs.bib}

\newcommand\labnr{4}
\newcommand\lab{Lab \labnr\ - Hash Tables}

\newcommand\uni{Technical University of Cluj-Napoca}
\newcommand\course{Data Structures \& Algorithms}

\newcommand\lvlez{$\bigstar$}
\newcommand\lvlmed{\lvlez\lvlez}
\newcommand\lvlhard{\lvlmed\lvlez}
\newcommand\lvlvhard{\lvlhard\lvlez}


\pagestyle{headandfoot}
\firstpageheader{}{}{}
\firstpagefootrule
\firstpageheadrule
\firstpagefooter{\sc\uni}{}{\sc\course, Lab \labnr}
\runningheader{\sc\uni}{}{\sc\course, Lab \labnr}
\runningheadrule
\runningfootrule
\runningfooter{}{\thepage}{}

\usepackage{listings}
\usepackage{xcolor}
\usepackage{svg}

\definecolor{codegreen}{rgb}{0,0.6,0}
\definecolor{codegray}{rgb}{0.5,0.5,0.5}
\definecolor{codepurple}{rgb}{0.58,0,0.82}
\definecolor{backcolour}{rgb}{0.95,0.95,0.92}

\lstdefinestyle{mystyle}{
    commentstyle=\color{codegreen},
    keywordstyle=\color{magenta},
    stringstyle=\color{codepurple},
    basicstyle=\ttfamily\footnotesize,
    breakatwhitespace=false,         
    breaklines=true,                 
    captionpos=b,                    
    keepspaces=true,                 
    showspaces=false,                
    showstringspaces=false,
    showtabs=false,                  
    tabsize=2
}
\lstset{style=mystyle}

\begin{document}
\begin{center}
    \vspace*{0cm}
    \bfseries\LARGE
    \lab
    \vspace*{1cm}
\end{center}


\begin{questions}
\question Implement a hash table that uses collision resolution by chaining using a data strucure similar to the following example:

\begin{lstlisting}[language=C]
typedef struct _NODE{
    int key;
    struct _NODE* next;
} NODE;

typedef struct {
    // the size of the table should be a prime number 
    NODE* v[17];
} HASH_TABLE;
\end{lstlisting}

\includesvg{diagrams/htchaining.svg}

Create a function for each of the following operations:
\begin{parts}
    \part Initialise the hashtable \lvlez
    \part Insert a value in the hashtable \lvlez
    \part Search a value in the hashtable \lvlez
    \part Delete a value from the hashtable \lvlmed
    \part Deallocate the hashtable (free the memory for the nodes and the table) \lvlez
\end{parts}

\question Implement a hash table that uses collision resolution by open addressing using a data strucure similar to the following example:

\begin{lstlisting}[language=C]
typedef struct HT_ENTRY{
    int key;
    char is_set;
    char is_gravestone;
} HT_ENTRY;

typedef struct {
    // a vector of HT_ENTRIES
    HT_ENTRY *v;
    // the size of the table should be a prime number 
    int ht_size;
    // optional, used to compute the load factor
    int nr_occupied;
} HASH_TABLE;
\end{lstlisting}

\includesvg{diagrams/htoa.svg}

\bigskip
\begin{parts}
    \part Implement the following operations by using linear probing (\verb|pos = (h(x) + i) % ht_size|)
    \begin{subparts}
        \subpart Initialise a table of a given size \lvlez
        \subpart Insert a value in the hashtable \lvlmed
        \subpart Search a value in the hashtable \lvlmed
        \subpart Delete a value from the hashtable \lvlmed
        \subpart Get the hash table's load factor \lvlmed
        \subpart Increase the table size and rehash the table if the load factor is above 0.7 \lvlhard
        \subpart Rehash the table if the number of gravestones is above 30\% of the hashtable's size \lvlhard
        \subpart Deallocate the table \lvlez
    \end{subparts}
    \part Implement the following operations by using quadratic probing (\verb|pos = (h(x) + i*i) % ht_size|)
    \begin{subparts}
        \subpart Initialise a table of a given size \lvlez
        \subpart Insert a value in the hashtable \lvlmed
        \subpart Search a value in the hashtable \lvlmed
        \subpart Delete a value from the hashtable \lvlmed
        \subpart Get the hash table's load factor \lvlmed
        \subpart Increase the table size and rehash the table if the load factor is above 0.7 \lvlhard
        \subpart Rehash the table if the number of gravestones is above 30\% of the hashtable's size \lvlhard
        \subpart Deallocate the table \lvlez
    \end{subparts}
    \part Implement the following operations by using double hashing (\verb|pos = (h(x)+ i*h2(x)) % ht_size|)
    \begin{subparts}
        \subpart Initialise a table of a given size \lvlez
        \subpart Insert a value in the hashtable \lvlmed
        \subpart Search a value in the hashtable \lvlmed
        \subpart Delete a value from the hashtable \lvlmed
        \subpart Get the hash table's load factor \lvlmed
        \subpart Increase the table size and rehash the table if the load factor is above 0.7 \lvlhard
        \subpart Rehash the table if the number of gravestones is above 30\% of the hashtable's size \lvlhard
        \subpart Deallocate the table \lvlez
    \end{subparts}
\end{parts}

\end{questions}

\bigskip
\textbf{Note:} Leave a comment with the text PB1, PB2.A.II, ... PB10 above every function that implements the respective lab task. (upper case text, no space between the text and the problem number)

\medskip
\printbibliography
\end{document}
